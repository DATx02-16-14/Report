% Copyright 2004 by Till Tantau <tantau@users.sourceforge.net>.
%
% In principle, this file can be redistributed and/or modified under
% the terms of the GNU Public License, version 2.
%
% However, this file is supposed to be a template to be modified
% for your own needs. For this reason, if you use this file as a
% template and not specifically distribute it as part of a another
% package/program, I grant the extra permission to freely copy and
% modify this file as you see fit and even to delete this copyright
% notice. 

\documentclass[10pt, compress, xcolor=table]{beamer}

\usepackage[utf8]{inputenc}
\usepackage{listings}
\usepackage{etoolbox}
\usepackage{booktabs}
\usepackage[scale=2]{ccicons}
\usepackage{minted}
\usemintedstyle{trac}


% There are many different themes available for Beamer. A comprehensive
% list with examples is given here:
% http://deic.uab.es/~iblanes/beamer_gallery/index_by_theme.html
% You can uncomment the themes below if you would like to use a different
% one:
%\usetheme{AnnArbor}
%\usetheme{Antibes}
\usetheme{m}
%\usetheme{Berkeley}
%\usetheme{Berlin}
%\usetheme{Boadilla}
%\usetheme{boxes}
%\usetheme{CambridgeUS}
%\usetheme{Copenhagen}
%\usetheme{Darmstadt}
%\usetheme{default}
%\usetheme{Frankfurt}
%\usetheme{Goettingen}
%\usetheme{Hannover}
%\usetheme{Ilmenau}
%\usetheme{JuanLesPins}
%\usetheme{Luebeck}
%\usetheme{Madrid}
%\usetheme{Malmoe}
%\usetheme{Marburg}
%\usetheme{Montpellier}
%\usetheme{PaloAlto}
%\usetheme{Pittsburgh}
%\usetheme{Rochester}
%\usetheme{Singapore}
%\usetheme{Szeged}
%\usetheme{Warsaw}
%\usecolortheme{beetle}
%\setbeamercolor{background canvas}{bg=white!98!black}
%\definecolor{sidebarcolor}{RGB}{70,44,110}
%\setbeamercolor{sidebar}{bg=sidebarcolor}


\title{The battle of paradigms}

% A subtitle is optional and this may be deleted
\subtitle{Functional vs Imperative}

\author{Joel~Gustafsson \and André~Samuelsson}
% - Give the names in the same order as the appear in the paper.
% - Use the \inst{?} command only if the authors have different
%   affiliation.

\institute[Chalmers University of Techonlogy] % (optional, but mostly needed)
{
  Department of Computer Science\\
  Chalmers University of Technology
}
% - Use the \inst command only if there are several affiliations.
% - Keep it simple, no one is interested in your street address.

\date{TDA 517, 2 May 2016}
% - Either use conference name or its abbreviation.
% - Not really informative to the audience, more for people (including
%   yourself) who are reading the slides online

\subject{Theoretical Computer Science}
% This is only inserted into the PDF information catalog. Can be left
% out. 

% If you have a file called "university-logo-filename.xxx", where xxx
% is a graphic format that can be processed by latex or pdflatex,
% resp., then you can add a logo as follows:

%\pgfdeclareimage[height=0.5cm]{university-logo}{haskelllogo}
%\logo{\pgfuseimage{university-logo}}

% Delete this, if you do not want the table of contents to pop up at
% the beginning of each subsection:
%\AtBeginSection[]
%{
%\ifnumcomp{\value{section}}{=}{1}{}
%  {
%  \begin{frame}<beamer>{Outline}
%    \tableofcontents[currentsection]
%  \end{frame}
%  }
%} 
  

% Let's get started
\begin{document}
\input{presentation-code.tex}

\begin{frame}[plain,noframenumbering]
  \titlepage
  \pagenumbering{gobble}
\end{frame}

%\begin{frame}{Outline}
%  \tableofcontents
  % You might wish to add the option [pausesections]
%\end{frame}

% Section and subsections will appear in the presentation overview
% and table of contents.
\section{Introduction}

\subsection{What are we going to talk about?}

\begin{frame}<beamer>{What are we going to talk about?}
    \tableofcontents
\end{frame}

\subsection{Why is this relevant?}
\begin{frame}{Why is functional programming relevant?}
    \begin{itemize}
        \item Imperative is dominant.
        \item Functional part of the future.
        \item Benefits of thinking functional.
        \item Another tool in your toolbox.
    \end{itemize}
\end{frame}

% You can reveal the parts of a slide one at a time
% with the \pause command:

\section{Differences}
%\subsection{Loop}
%\begin{frame}{Example: Loop}
%    \begin{block}{Imperative (Java)}
%           \makejavafor
%    \end{block}
%    \begin{block}{Declarative (Haskell)}
%        \makehaskellfor
%    \end{block}
%\end{frame}
\subsection{Eratosthenes sieve}

\begin{frame}{Example: Eratosthenes sieve}
    \begin{block}{Imperative (Java)}
        \makejavasieve
    \end{block}
\end{frame}

\begin{frame}{Example: Eratosthenes sieve}
    \begin{block}{Declarative (Haskell)}
        \makehaskellsieve
    \end{block}
\end{frame}

\subsection{Important differences}
\begin{frame}{Important differences}

\begin{table}[]
\small
\centering
\begin{tabular}{|l|l|l|}
\hline
\rowcolor[HTML]{C0C0C0} 
{\color[HTML]{000000} Characteristic} & {\color[HTML]{000000} Imperative}                                                     & {\color[HTML]{000000} Functional}                                                                        \\ \hline
Flow control                          & \begin{tabular}[c]{@{}l@{}}Loops, if-else,\\ function calls.\end{tabular}             & \begin{tabular}[c]{@{}l@{}}function calls,\\ recursion.\end{tabular}                                     \\ \hline
Developer focus                       & \begin{tabular}[c]{@{}l@{}}Updating and Tracking state.\\ Perform tasks.\end{tabular} & \begin{tabular}[c]{@{}l@{}}Information flow,\\ transformations on data\end{tabular}                      \\ \hline
%Execution order                       & \begin{tabular}[c]{@{}l@{}}Important since it\\ can effect a state.\end{tabular}      & \begin{tabular}[c]{@{}l@{}}No side effects in\\ functions which\\ encourage lazy evaluation\end{tabular} \\ \hline
\end{tabular}
\end{table}
    
\end{frame}

\section{Benefits}
\subsection{When is functional beneficial}
\begin{frame}{When is functional beneficial}
    \begin{itemize}
        \item{
            Abstraction
        }
        \item {
            Mathematical reasoning
        }
        \item{
            Programmer productivity\\
            C++ 962\\
            Haskell 173
        }
    \end{itemize}
\end{frame}
\section{Disadvantages}
\subsection{When is it not beneficial}
\begin{frame}{When is it not beneficial}
    \begin{itemize}
        \item {
            Performance
        }
        \item {
            Complexity
        }
    \end{itemize}
    \pause
    \begin{example}
        \makehaskellcomplex
    \end{example}
\end{frame}
\subsection{Why you should learn it regardless}
\begin{frame}{Why you should learn it regardless}
    \begin{itemize}
        \item{
            Complexity becomes an advantage
        }
        \item {
            Easy to reason about problems
        }
        \item {
            It is on the rise
        }
    \end{itemize}
\end{frame}


% Placing a * after \section means it will not show in the
% outline or table of contents.
\section{Summary}

\begin{frame}{Summary}
  \begin{itemize}
  \item 
    We showed some example code to point out differences.
  \item
    The functional paradigm is a huge part of the \alert{future} in programming.
  \item
    It helps with programmer \alert{productivity}.
  \item
    But it can be \alert{slower}, than some other languages.
  \end{itemize}
  
  \begin{itemize}
  \item
    Outlook
    \begin{itemize}
    \item
      Optimising Haskell code to be as efficient as C++?
    \item
      Establish any learning difficulties.
    \end{itemize}
  \end{itemize}
\end{frame}


\begin{frame}{Questions?}
    \begin{block}{Any questions?}
    \end{block}
  
\end{frame}

% All of the following is optional and typically not needed. 
\iffalse
\appendix
\section<presentation>*{\appendixname}
\subsection<presentation>*{For Further Reading}

\begin{frame}
  \frametitle<presentation>{For Further Reading}
    
  \begin{thebibliography}{10}
    
  \beamertemplatebookbibitems
  % Start with overview books.

  \bibitem{Author1990}
    A.~Author.
    \newblock {\em Handbook of Everything}.
    \newblock Some Press, 1990.
 
    
  \beamertemplatearticlebibitems
  % Followed by interesting articles. Keep the list short. 

  \bibitem{Someone2000}
    S.~Someone.
    \newblock On this and that.
    \newblock {\em Journal of This and That}, 2(1):50--100,
    2000.
  \end{thebibliography}
\end{frame}
\fi

\end{document}


